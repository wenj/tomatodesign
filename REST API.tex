% !Mode:: "TeX:UTF-8"
\documentclass{article}
\usepackage[hyperref, UTF8]{ctex}
\usepackage[dvipsnames]{xcolor}
\usepackage{geometry}
\usepackage{amsmath}
\usepackage{amsfonts}
\usepackage{listings}
\usepackage{pgfplotstable}
\usepackage{pgfplots}
\usepackage{fontspec}
\usepackage{booktabs} % 表格上的不同横线
%\setmainfont{Consolas} %设置英文字体

\definecolor{mygreen}{rgb}{0,0.6,0}
\definecolor{mygray}{rgb}{0.5,0.5,0.5}
\definecolor{mymauve}{rgb}{0.58,0,0.82}
\lstset{ %
    backgroundcolor=\color{white},   % choose the background color
    basicstyle=\footnotesize\ttfamily,        % size of fonts used for the code
    columns=fullflexible,
    breaklines=true,                 % automatic line breaking only at whitespace
    captionpos=b,                    % sets the caption-position to bottom
    tabsize=4,
    backgroundcolor=\color[RGB]{245,245,244},            % 设定背景颜色
    commentstyle=\color{mygreen},    % comment style
    escapeinside={\%*}{*)},          % if you want to add LaTeX within your code
    keywordstyle=\color{blue},       % keyword style
    stringstyle=\color{mymauve}\ttfamily,     % string literal style
    showstringspaces=false,                % 不显示字符串中的空格
    frame=none,
    rulesepcolor=\color{red!20!green!20!blue!20},
    % identifierstyle=\color{red},
    language=c++,
}

% 设置hyperlink的颜色
\newcommand\myshade{85}
\colorlet{mylinkcolor}{violet}
\colorlet{mycitecolor}{YellowOrange}
\colorlet{myurlcolor}{Aquamarine}

\hypersetup{
  linkcolor  = mylinkcolor!\myshade!black,
  citecolor  = mycitecolor!\myshade!black,
  urlcolor   = myurlcolor!\myshade!black,
  colorlinks = true,
}


\title{ASDAN商业竞赛项目 API设计}
\author{Tomato小组}

\begin{document}

\maketitle

\tableofcontents

\newpage

\begin{enumerate}
    \item Client API Endpoints: 为特定用户提供相关信息。
    \item Admin API Endpoints: 为管理员提供相关信息。
    \item Administrative Endpoints: 登录和交易时使用。
    \item Utility Endpoints: 询问系统相关信息。
\end{enumerate}

\section{一些修改及说明}
10.29:修改了时间的表示方法,目前按照ISO-8601格式,具体请参照这篇文章:\url{https://stackoverflow.com/questions/19013562/java-dates-and-standard-formats}。

\begin{lstlisting}
"yyyy-MM-dd'T'HH:mm:ss.SSS'Z'"
\end{lstlisting}

\section{Client API Endpoints}

为特定用户提供相关信息。

\subsection{Get Information}
输入ID,获得与这一ID相关的用户的配置信息。用户有默认头像。如果比赛还没有开始,则rank为0。


\subsubsection*{Request}
\begin{lstlisting}
GET /api/client/info/id={id}

Host: localhost:8080
Auth:
Content-type: application/json
Accept: application/json
\end{lstlisting}
\subsubsection*{Returns}
\begin{lstlisting}
HTTP 200 OK

{
   "memberList":
    [
        "member1":"wangmz",
    	"member2":"songsh",
    	"member3":"wtf"
    ],
    "username": "team1",
    "id":3,
    "rank": 1,
    
    "gameStatus": "auction"("not_start", "auction_not_record", "auction_recorded", "trade", "rest", "end")
    "round": 0/1/2/3   (第(round+1)轮)
    "timeLeft":300(s)
}
\end{lstlisting}
\subsubsection*{Error}
\begin{lstlisting}
HTTP 404 NOT FOUND
\end{lstlisting}






\subsection{Get Avatar}
获得用户头像

\subsubsection*{Request}
\begin{lstlisting}
GET /api/client/info/getAvatar/id={id}

Host: localhost:8080
Auth:
Content-type: application/json
Accept: application/json
\end{lstlisting}
\subsubsection*{Returns}
\begin{lstlisting}
HTTP 200 OK

HashMap
{
    "fileoriginalsize", size;
         "contenttype", contentType;
         "base64", new String(Base64Utils.encode(IOUtils.toByteArray(stream)));
}
\end{lstlisting}
\subsubsection*{Error}
\begin{lstlisting}
HTTP 404 NOT FOUND
\end{lstlisting}







\subsection{Get Property}
输入ID,获得与这一ID相关的用户的财产信息。包括机器的使用情况和材料的价格。
\subsubsection*{Request}
\begin{lstlisting}
GET /api/client/property/id={id}

Host: localhost:8080
Auth:
Content-type: application/json
Accept: application/json
\end{lstlisting}

\subsubsection*{Returns}
\begin{lstlisting}
HTTP 200 OK

{
   "wealth": 3000,
   "machine":
    [
        {
            "id": 0073,
            "type": "Cement",
            "left": 3
            "lock":false
        },
        {
            "id": 0793,
            "type": "Brick",
            "left": 0
            "lock":true(正处于出售中的机器和材料 lock == true)
        },
        {
            "id": 8765,
            "type": "Wood",
            "left": 2
            "lock":false
        }
    ],
    "material":
    [
        {
            "type": "Wood",
            "price": 10,
            "number": 20,
            "lock":true
        },
        {
           "type": "Brick",
            "price": 20,
            "number": 0,
            "lock":false
        },
        {
          "type": "Cement",
            "price": 80,
            "number": 150,
            "lock":false
        }
    ]
}
\end{lstlisting}

\subsubsection*{Error}
\begin{lstlisting}
HTTP 404 NOT FOUND
\end{lstlisting}






\subsection{Update Avatar}
向服务器发送用户更改过的头像。
(其它队友获取新头像直接调用get Avatar)
\subsubsection*{Request}
\begin{lstlisting}
POST /api/client/info/updateAvatar/id={id}

Host: localhost:8080
MultipartHttpServletRequest request
\end{lstlisting}

\subsubsection*{Returns}
\begin{lstlisting}
HTTP 200 OK

HashMap
{
    "fileoriginalsize", size;
         "contenttype", contentType;
         "base64", new String(Base64Utils.encode(IOUtils.toByteArray(stream)));
}
\end{lstlisting}
\subsubsection*{Error}
\begin{lstlisting}
HTTP 404 NOT FOUND
\end{lstlisting}



\subsection{Get All User}
get 所有队伍,(除了发送消息的队伍),用来发sell Request时进行选择
\subsubsection*{Request}
\begin{lstlisting}
GET /api/client/getAllUser/id={id}

Host: localhost:8080
Auth:
Content-type: application/json
Accept: application/json

\end{lstlisting}
\subsubsection*{Returns}
\begin{lstlisting}
HTTP 200 OK

{
    [
        {
        "teamId": 889,
            "username": "dddd",
        },
        {
               "teamId": 999,
            "username": "ddfadf",
        },
    ]
}
\end{lstlisting}
\subsubsection*{Error}
\begin{lstlisting}
HTTP 404 NOT FOUND
\end{lstlisting}



\subsection{交易:发出出售请求,buyer监听}
\begin{lstlisting}
 MessageMapping: /api/client/property/sellerId={sellerId}/buyerId={buyerId}

 SendTo: /api/client/property/buyerId={buyerId}
 
 EndPoint:  http://127.0.0.1:8090/trade

 Get JSon Pattern:(卖方发送)
 {
    "tradeId": '',
    "sellerId":sellerId,
     "buyerId":buyerId,
     "buyer":"TOMATO"
     "typeOrMachineID":"9987"  ("Wood" "Cement" "Brick" OR machineID)
     "price":300,
     "number":7
     "seller":"Rua"
}
Send JSon Pattern:(发给买方)
 {
    "tradeId": tradeId,
    "sellerId": sellerId,
     "buyerId": buyerId,
     "buyer": "TOMATO"
     "typeOrMachineID": "9987"  ("Wood" "Cement" "Brick" OR machineID)
     "price": 300,
     "number": 7
     "seller": "Rua"
 }

\end{lstlisting}






\subsection{交易结束 给卖家转发账单和现有资产}
\begin{lstlisting}
 MessageMapping: /api/client/tradeFinish/sellerId={sellerId}

 SendTo: /api/client/tradeFinish/id={sellerId}
 
 EndPoint:  http://127.0.0.1:8090/tradeFinish
 
 Get JSon Pattern:
 {
    "tradeId":tradeId,
    "sellerId":sellerId,
    "buyerId":buyerId,
    "buyer":"TOMATO"
    "typeOrMachineID":"9987"  ("Wood" "Cement" "Brick" OR machineID)
    "price":300,
    "number":7
    "seller":"Rua"
     
     "isAccept":true
}
 
Send JSon Pattern:
{
    "reply":
    {
    "tradeId":tradeId,
    "sellerId":sellerId,
    "buyerId":buyerId,
    "buyer":"TOMATO"
    "typeOrMachineID":"9987"  ("Wood" "Cement" "Brick" OR machineID)
    "price":300,
    "number":7
    "seller":"Rua"
    
    "isAccept":true
     }
     
   
   "propertyList":
   {
   	"wealth":1000,
   	"machineList":
    	[
        {
            "id": 0073,
            "type": "Wood",
            "left": 3
            "lock":false
        },
        ],
    	"materialList":
    	[
        {
            "type": "Brick",
            "price": 20,
            "number": 0,
            "lock":false
        },
    	]
    }
}
\end{lstlisting}






\subsection{交易结束 给队友转发账单和现有资产}
\begin{lstlisting}
 MessageMapping: /api/client/tradeFinish/buyerId={buyerId}

 SendTo: /api/client/tradeFinish/id={buyerId}
 
 EndPoint:  http://127.0.0.1:8090/tradeFinish
 
 Get JSon Pattern:
 {
    "tradeId":tradeId,
    "sellerId":sellerId,
    "buyerId":buyerId,
    "buyer":"TOMATO"
    "typeOrMachineID":"9987"  ("Wood" "Cement" "Brick" OR machineID)
    "price":300,
    "number":7
    "seller":"Rua"
     
     "isAccept":true
}
 
Send JSon Pattern:
{
    "reply":
    {
    "tradeId":tradeId,
    "sellerId":sellerId,
    "buyerId":buyerId,
    "buyer":"TOMATO"
    "typeOrMachineID":"9987"  ("Wood" "Cement" "Brick" OR machineID)
    "price":300,
    "number":7
    "seller":"Rua"
    
    "isAccept":true
     }
     
   
   "propertyList":
   {
   	"wealth":1000,
   	"machineList":
    	[
        {
            "id": 0073,
            "type": "Wood",
            "left": 3
            "lock":false
        },
    	],
    	"materialList":
    	[
        {
            "type": "Brick",
            "price": 20,
            "number": 0,
            "lock":false
        },
    	]
    }
}
\end{lstlisting}



\subsection{监听比赛状态改变}
\begin{lstlisting}
 MessageMapping: /api/admin/competition/status/id={id}
 
 SendTo: /api/client/CompetitionStatus/id={id}

EndPoint:  http://127.0.0.1:8090/competitionStatus

Send JSon Pattern:
 {
    	"gameStatus": "auction"("not_start", "auction_not_record", "auction_recorded", "trade", "rest", "end")
        "round": 0/1/2/3   (第(round+1)轮)
        "timeLeft":227(s)

 }

 Get JSON Pattern:
 {
    "gameStatus": "auction"("not_start", "auction_not_record", "auction_recorded", "trade", "rest", 
    "round": 0/1/2/3   (从0开始)
 }
\end{lstlisting}



\subsection{监听produce后资产的改变}
\begin{lstlisting}
 MessageMapping: /api/client/ListenProperty/id=3
 
 SendTo: /api/client/ListenProperty/receive/id=3

EndPoint:  http://127.0.0.1:8090/listenProperty

Send JSon Pattern:
{
   "wealth":1000,
   "machine":
    [
        {
            "id": 0073,
            "type": "Wood",
            "left": 3
            "lock":false
        },
        {
            "id": 0793,
            "type": "Brick",
            "left": 0
            "lock":true
        },
        {
            "id": 8765,
            "type": "Cement",
            "left": 2
            "lock":false
        }
    ],
    "material":
    [
        {
           "type": "Wood",
            "price": 10,
            "number": 20,
            "lock":false
        },
        {
           "type": "Brick",
            "price": 20,
            "number": 0,
            "lock":false
        },
        {
           "type": "Cement",
            "price": 80,
            "number": 150,
            "lock":false
        }
    ]
}

Get JSON Pattern:
{
   "id":2,
   "times":1,
}
}
\end{lstlisting}




\subsection{Get Trade History}
交易历史信息。在发订单的时候客户端手动更新History。
\subsubsection*{Request}
\begin{lstlisting}
GET /api/client/tradehHistory/id={id}

Host: localhost:8080
Auth:
Content-type: application/json
Accept: application/json
\end{lstlisting}
\subsubsection*{Returns}
\begin{lstlisting}
HTTP 200 OK
        [
            {
            "time":   "yyyy-MM-dd'T'HH:mm:ss.SSS'Z'",
            "target": "team1",
            "action": "sell",
            "content": "wood",
            "price": 10,
            "number": 2
            "status": 1,(完成)
            "tradeId":44,
            "buyerId":9
            },

            {
            "time":   "yyyy-MM-dd'T'HH:mm:ss.SSS'Z'",
            "target": "team1",
             "action": "buy",
            "content": "1234" (machine.id ==1234)
            "price": 10,
            "number": 1     (只能是1)
            "status": 0,    (正在进行)
            "tradeId":44,
            "buyerId":9
            },

            {
            "time":   "yyyy-MM-dd'T'HH:mm:ss.SSS'Z'",
            "target": "team1",
            "action": "buy",
            "content": "6666"    (machine.id ==1234)
            "price": 10,
            "number": 1     (只能是1)
            "status": -1,   (失败)
            "tradeId":44,
            "buyerId":9,
            }
        ]\end{lstlisting}
\subsubsection*{Error}
\begin{lstlisting}
HTTP 404 NOT FOUND
\end{lstlisting}





\subsection{Get Produce History}
生产历史信息。
\subsubsection*{Request}
\begin{lstlisting}
GET /api/client/produceHistory/id={id}

Host: localhost:8080
Auth:
Content-type: application/json
Accept: application/json
\end{lstlisting}
\subsubsection*{Returns}
\begin{lstlisting}
HTTP 200 OK
        [
            {
            "time":   "yyyy-MM-dd'T'HH:mm:ss.SSS'Z'",
            "machineId":9987
            "content": "Brick",
            "price": 10,
            "number": 2
            },
            {
            "time":   "yyyy-MM-dd'T'HH:mm:ss.SSS'Z'",
            "machineId":3457
            "content": "Wood",
            "price": 10,
            "number": 2
            },
            {
            "time":   "yyyy-MM-dd'T'HH:mm:ss.SSS'Z'",
            "machineId":5777
            "content": "Cement",
            "price": 10,
            "number": 2
            },
        ]\end{lstlisting}
\subsubsection*{Error}
\begin{lstlisting}
HTTP 404 NOT FOUND
\end{lstlisting}




\subsection{撤销,卖方监听}
\begin{lstlisting}
 MessageMapping: /api/client/undo/sendToSeller/sellerId={sellerId}/buyerId={buyerId}

 SendTo: /api/client/receiveUndo/id={sellerId}
 
 EndPoint:  http://127.0.0.1:8090/undo

 Get JSon Pattern:(卖方发送)
 {
    "tradeId": 77,
 }
Send JSon Pattern:(发给卖方)
 {
    "request":
    {
    "tradeId":77,
    "sellerId":sellerId,
    "buyerId":buyerId,
    "buyer":"TOMATO"
    "typeOrMachineID":"9987"  ("Wood" "Cement" "Brick" OR machineID)
    "price":300,
    "number":7
    "seller":"Rua"
     }
     
   "propertyList":
   {
   	"wealth":1000,
   	"machineList":
    	[
        {
            "id": 0073,
            "type": "Wood",
            "left": 3
            "lock":false
        },
    	],
    	"materialList":
    	[
        {
            "type": "Brick",
            "price": 20,
            "number": 0,
            "lock":false
        },
    	]
    }
 }

\end{lstlisting}





\subsection{撤销,买方监听}
\begin{lstlisting}
 MessageMapping: /api/client/undo/sendToBuyer/sellerId={sellerId}/buyerId={buyerId}

 SendTo: /api/client/receiveUndo/id={buyerId}
 
 EndPoint:  http://127.0.0.1:8090/undo

 Get JSon Pattern:(卖方发送)
 {
    "tradeId": 77,
 }
Send JSon Pattern:(发给买方)
 {
    "request":
    {
    "tradeId":77,
    "sellerId":sellerId,
    "buyerId":buyerId,
    "buyer":"TOMATO"
    "typeOrMachineID":"9987"  ("Wood" "Cement" "Brick" OR machineID)
    "price":300,
    "number":7
    "seller":"Rua"
     }
     
   "propertyList":
   {
   	"wealth":1000,
   	"machineList":
    	[
        {
            "id": 0073,
            "type": "Wood",
            "left": 3
            "lock":false
        },
    	],
    	"materialList":
    	[
        {
            "type": "Brick",
            "price": 20,
            "number": 0,
            "lock":false
        },
    	]
    }
 }

\end{lstlisting}







\section{Admin API Endpoints}
\subsection{Get All Competitions}

列出全部比赛。

Status是"not\_start", "auction\_not\_record", "auction\_recorded", "trade", "rest", "end"之一。

\subsubsection*{Request}
\begin{lstlisting}
GET /api/admin/competition/getall

Host: localhost:8080
Auth:
Content-type: application/json
Accept: application/json
\end{lstlisting}

\subsubsection*{Returns}
\begin{lstlisting}
HTTP 200 OK

[
    {
        "id": "competiton1_id",
        "username": "competition1",
        "status": "auction"
    },
    {
        "id": "competiton2_id",
        "username": "competition2",
        "status": "end"
    }
]
\end{lstlisting}

\subsubsection*{Error}
\begin{lstlisting}
HTTP 204 NO CONTENT
\end{lstlisting}

\subsection{Create Competition}
新建一场比赛。注意,底层也要生成机器的id。注意每场比赛的基本配置(比赛名称,参赛人数)只能创建一次,不能修改。

\subsubsection*{Request}
\begin{lstlisting}
POST /api/admin/competition/new

Host: localhost:8080
Auth:
Content-type: application/json
Accept: application/json

{
    "username": "competition_username",
    "round": 2,
    "startWealth": 1000,

    "teamNum": 2,
    "participantNum": 3,
    "team":
    [
        {
            "username": "team1",
            "participant": ["mem11", "mem12", "mem13"],
            "password": "111111",
        },
        {
            "username": "team2",
            "participant": ["mem21", "mem22", "mem23"],
            "password": "222222",
        }
    ]

    "roundParameter":
    [
        {
            "machineStartPrice": [300, 350, 400],
            "machineNum": [1, 1, 1],
            "materialProduceCost": [10, 20, 30],
            "materialPerHouse": [10, 10, 10],
            "time": 900,
        },
        {
            "machineStartPrice": [300, 350, 400],
            "machineNum": [1, 1, 1],
            "materialProduceCost": [10, 20, 30],
            "materialPerHouse": [10, 10, 10],
            "time": 900,
        }
    ]
}
\end{lstlisting}

\subsubsection*{Returns}
\begin{lstlisting}
HTTP 201 CREATED

\end{lstlisting}

\subsubsection*{Error}
\begin{lstlisting}
HTTP 404 NOT FOUND
{
    "error":"Unable to delete. Competition with id xxx not found."
}
\end{lstlisting}


\subsection{Delete Competition By ID}

通过ID删除比赛。

\subsubsection*{Request}
\begin{lstlisting}
DELETE /api/admin/competiton/id={competition_id}

Host: localhost:8080
Auth:
Content-type: application/json
Accept: application/json
\end{lstlisting}

\subsubsection*{Returns}
\begin{lstlisting}
HTTP 200 OK
[
    {
        "id": "competiton2_id",
        "username": "competition2",
        "status": "end"
    }
]
\end{lstlisting}

\subsubsection*{Error}
\begin{lstlisting}
{
    "error":"Unable to delete. Competition with id xxx not found."
}
\end{lstlisting}

\subsection{Update Competition Status}

需要进入下一环节时,管理员端会向服务器发送更新比赛状态的请求,服务器返回当前比赛信息以便管理员端更新到最新的比赛状态。

\subsubsection*{Web Socket}
\begin{lstlisting}
MessageMapping: /api/admin/status/update/id=3
 
SendTo: /api/admin/status/id=3

EndPoint:  http://127.0.0.1:8090/competitionStatus

Send JSON Pattern:
{
    "status": "auction"("not_start", "auction_not_record", "auction_recorded", "trade", "rest", "end")
    "round": 0/1/2/3 
    "timeLeft":227(s)
 }

Get JSON Pattern: 
{
    "status": "auction"("not_start", "auction_not_record", "auction_recorded", "trade", "rest", "end")
    "round": 0/1/2/3
}
\end{lstlisting}

\subsubsection*{Error}
\begin{lstlisting}
HTTP 404 NOT FOUND
{
    "error":"Unable to update. Competition with id xxx not found"
}
\end{lstlisting}

\subsection{Get Auction Machine}

获得某场比赛某一轮拍卖机器的初始信息。

\subsubsection*{Request}
\begin{lstlisting}
GET /api/admin/competition/auction/id={id}/round={round}

Host: localhost:8080
Auth:
Content-type: application/json
Accept: application/json
\end{lstlisting}

\subsubsection*{Returns}
\begin{lstlisting}
HTTP 200 OK
[
    {
        "machineId": "machine1",
        "type": "Wood",
        "startPrice": 200,
    },
    {
        "machineId": "machine2",
        "type": "Brick",
        "startPrice": 300,
    },
    {
        "machineId": "machine3",
        "type": "Cement",
        "startPrice": 400,
    }
]
\end{lstlisting}

\subsubsection*{Error}
\begin{lstlisting}
HTTP 404 NOT FOUND
{
    "error": "Competition with id xxx not found." (or Competition with id xxx does not have round xxx)
}
\end{lstlisting}

\subsection{Record Auction Result}

登记某场比赛某一轮的拍卖结果。

\subsubsection*{Request}
\begin{lstlisting}
POST /api/admin/competition/record/id={id}/round={round}

Host: localhost:8080
Auth:
Content-type: application/json
Accept: application/json
\end{lstlisting}

\subsubsection*{Returns}
\begin{lstlisting}
HTTP 200 OK
[
    {
        "machineId": "machine1",
        "teamId": "team1",
        "price": 2000,
    },
    {
        "machineId": "machine2",
        "teamId": "brick",
        "price": 3000,
    },
    {
        "machineId": "machine3",
        "teamId": "cement",
        "price": 4000,
    }
]
\end{lstlisting}

\subsubsection*{Error}
\begin{lstlisting}
HTTP 404 NOT FOUND
{
    "error": "Competition with id xxx not found." (or Competition with id xxx does not have round xxx)
}
\end{lstlisting}


\subsection{Get Competition Property}

从服务器按id获取某一比赛的各种属性。如果该比赛的属性尚未被设置,则该项为空。属性包括名称、比赛轮数(如果比赛已开始,则不能删除已开始或结束的轮)、比赛各项参数(不能修改已开始或结束的轮的参数)、机器的id等等。

\subsubsection*{Request}
\begin{lstlisting}
GET /api/admin/competition/property/id={id}

Host: localhost:8080
Auth:
Content-type: application/json
Accept: application/json
\end{lstlisting}

\subsubsection*{Returns}
\begin{lstlisting}
HTTP 200 OK
{
    "id": "competition_id",
    "username": "competition_username",
    "status": "not started",
    "teamNum": 1,
    "participantNum": 2,
    "team":
    [
        {
            "username": "team1",
            "participant": ["member1", "member2", "member2"],
            "password": "password",
        }
    ]
    "round": 1,
    "startWealth": 1000,
    "roundParameter":
    [
        {
            "machineStartPrice": [300, 350, 400],
            "machineNum": [1, 1, 1],
            "materialProduceCost": [10, 20, 30],
            "materialPerHouse": [10, 10, 10],
            "time": 900,
        }
    ]
}

\end{lstlisting}

\subsubsection*{Error}
\begin{lstlisting}
HTTP 404 NOT FOUND
{
    "error": "Competition with id xxx not found."
}
\end{lstlisting}

\subsection{Update Competition Property}

更新比赛的各种属性。属性包括名称、比赛轮数(如果比赛已开始,则不能更改)、比赛各项参数(不能修改已开始或结束的轮的参数)。

\subsubsection*{Request}
\begin{lstlisting}
PUT /api/admin/competition/property/id={id}

Host: localhost:8080
Auth:
Content-type: application/json
Accept: application/json

{
    "round": 2,
    "startWealth": 1000,
    "round_parameter":
    [
        {
            "machineStartPrice": [300, 350, 400],
            "machineNum": [1, 1, 1],
            "materialProduceCost": [10, 20, 30],
            "materialPerHouse": [10, 10, 10],
            "time": 900,
        },
        {
            "machineStartPrice": [300, 350, 400],
            "machineNum": [1, 1, 1],
            "materialProduceCost": [10, 20, 30],
            "materialPerHouse": [10, 10, 10],
            "time": 900,
        }
    ]
}
\end{lstlisting}

\subsubsection*{Returns}
\begin{lstlisting}
HTTP 201 CREATED
\end{lstlisting}

\subsubsection*{Error}
\begin{lstlisting}
HTTP 404 NOT FOUND
{
    "error": "Competition with id xxx not found."
}
\end{lstlisting}

\begin{lstlisting}
HTTP 400 INVALID REQUEST
{
    "error": "Cannot update competition id xxx with given changes."
}
\end{lstlisting}

\subsection{Get Competition Information}
获取当前比赛信息,包括队伍的数量、资产、交易记录、机器的使用情况等。

\subsubsection*{Request}
\begin{lstlisting}
GET /api/admin/competition/info/id={id}

Host: localhost:8080
Auth:
Content-type: application/json
Accept: application/json
\end{lstlisting}

\subsubsection*{Returns}
\begin{lstlisting}
HTTP 200 OK
{
    "id": "competition_id",
    "username": "competition_username",
    "status": "not_start",
    "round": 2,
    "presentRound": 0,
    "teamInfo":
    [
        {
            "id": "id1",
            "username": "username1",
            "house": 5,
            "wealth": 100,
            "material": [30, 40, 50],
            "machine":
            [
                 {
                    "id": "machine1_id",
                    "type": "type1",
                    "left": 3
                 },
                 {
                    "id": "machine2_id",
                    "type": "type2",
                    "left": 2
                 }
            ]
        },
        {
            "id": "id2",
            "username": "username2",
            "house": 5,
            "wealth": 100,
            "material": [30, 40, 50],
            "machine":
            [
                 {
                    "id": "machine1_id",
                    "type": "type1",
                    "left": 3
                 },
                 {
                    "id": "machine2_id",
                    "type": "type2",
                    "left": 2
                 }
            ]
        }
    ],
    "trade_history":
    [
        {
            "time": "yyyy-MM-dd'T'HH:mm:ss.SSS'Z'",
            "sell": "team_id1",
            "buy": "team_id2",
            "content": {"wood": 1},
            "price": 10
        },
         {
            "time": "yyyy-MM-dd'T'HH:mm:ss.SSS'Z'",
            "sell": "team_id1",
            "buy": "team_id2",
            "content": {"machine_wood": 1},
            "price": 20
        }
    ]
}

\end{lstlisting}
content中是wood,brick,cement,machine_wood,machine_brick,machine_cement中的一个。

\subsubsection*{Error}
\begin{lstlisting}
HTTP 404 NOT FOUND
{
    "error": "Competition with id 1 not found."
}
\end{lstlisting}

\subsection{Record machine owner}
向服务器发送对比赛的更新信息。增加机器、分配财产之类的。

\subsubsection*{Request}
\begin{lstlisting}
PUT /api/admin/competition/info/id={id}

Host: localhost:8080
Auth:
Content-type: application/json
Accept: application/json

{
    "round": 2,
    "present_round": 0,
    "team_info":
    [
        {
            "id": "id1",
            "wealth": "100",
            "machine":
            [
                 {
                    "id": "machine1_id",
                    "left": 3
                 },
                 {
                    "id": "machine2_id",
                    "left": 2
                 }
            ]
        },
        {
            "id": "id2",
            "wealth": "100",
            "machine":
            [
                 {
                    "id": "machine1_id",
                    "left": 3
                 },
                 {
                    "id": "machine2_id",
                    "left": 2
                 }
            ]
        }
    ]
}
\end{lstlisting}

\subsubsection*{Returns}
\begin{lstlisting}
HTTP 200 OK
{
    "id": "competition_id",
    "username": "competition_username",
    "status": "not started",
    "round": "2",
    "present_round": "0",
    "team_info":
    [
        {
            "id": "id1",
            "wealth": 100,
            "material": [30, 40, 50],
            "machine":
            [
                 {
                    "id": "machine1_id",
                    "type": "type1",
                    "left": 3
                 },
                 {
                    "id": "machine2_id",
                    "type": "type2",
                    "left": 2
                 }
            ]
        },
        {
            "id": "id2",
            "wealth": 100,
            "material": [30, 40, 50],
            "machine":
            [
                 {
                    "id": "machine1_id",
                    "type": "type1",
                    "left": 3
                 },
                 {
                    "id": "machine2_id",
                    "type": "type2",
                    "left": 2
                 }
            ]
        }
    ],
    "trade_history":
    [
        {
            "time": "hh:MM:ss",
            "sell": "team_id1",
            "buy": "team_id2",
            "content": {"wood": 1},
            "price": 10
        },
         {
            "time": "hh:MM:ss",
            "sell": "team_id1",
            "buy": "team_id2",
            "content": {"machine_wood": 1},
            "price": 20
        }
    ]
}

\end{lstlisting}
content中是wood,brick,cement,machine_wood,machine_brick,machine_cement中的一个。

\subsubsection*{Error}
\begin{lstlisting}
HTTP 404 NOT FOUND
{
    "error": "Competition with id xxx not found."
}
\end{lstlisting}

\begin{lstlisting}
HTTP 400 INVALID REQUEST
{
    "error": "Cannot update competition id xxx with given information."
}
\end{lstlisting}

\section{Administrative Endpoints}
登录和交易时使用。

\subsection{Login Admin}
管理员登录。

\subsubsection*{Request}
\begin{lstlisting}
POST /api/admin/login

Host: localhost:8080
Auth:
Content-type: application/json
Accept: application/json

{
    "username": "admin",
    "password": "admin",
}
\end{lstlisting}

\subsubsection*{Returns}
\begin{lstlisting}
HTTP 200 OK
{
    "username": "admin",
    "token": "1283091828021803120",
}

\end{lstlisting}

\subsubsection*{Error}
\begin{lstlisting}
HTTP 401 NOTAUTHORIZED
{
    "error": "Admin with username admin doesn\'t exist or password is wrong."
}
\end{lstlisting}

\subsection{Change Admin Password}
更改管理员密码。

\subsubsection*{Request}
\begin{lstlisting}
POST /api/admin/update

Host: localhost:8080
Auth:
Content-type: application/json
Accept: application/json

{
    "username": "admin",
    "prePassword": "admin",
    "newPassword": "newpwd",
}
\end{lstlisting}

\subsubsection*{Returns}
\begin{lstlisting}
HTTP 200 OK

\end{lstlisting}

\subsubsection*{Error}
\begin{lstlisting}
HTTP 401 NOTAUTHORIZED
{
    "error": "Wrong password of admin."
}
\end{lstlisting}

\subsection{Login Client}
用户登录。

\subsubsection*{Request}
\begin{lstlisting}
POST /api/client/login

Host: localhost:8080
Auth:
Content-type: application/json
Accept: application/json

{
    "username": "client",
    "password": "client",
}
\end{lstlisting}

\subsubsection*{Returns}
\begin{lstlisting}
HTTP 200 OK
{
    "username": "client",
     "id": "3",
    "token": "1283091828021803120",
}

\end{lstlisting}

\subsubsection*{Error}
\begin{lstlisting}
HTTP 401 NOTAUTHORIZED
{
    "error": "Client with userusername admin doesn\'t exist or password is wrong."
}
\end{lstlisting}

\section{Utility Endpoints}
询问系统相关信息。

\end{document}
