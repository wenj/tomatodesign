% !Mode:: "TeX:UTF-8"
\documentclass{article}
\usepackage[hyperref, UTF8]{ctex}
\usepackage[dvipsnames]{xcolor}
\usepackage{geometry}
\usepackage{amsmath}
\usepackage{amsfonts}
\usepackage{listings}
\usepackage{pgfplotstable}
\usepackage{pgfplots}
\usepackage{fontspec}
\usepackage{booktabs} % 表格上的不同横线
\setmonofont[Mapping={}]{Consolas}	%英文引号之类的正常显示,相当于设置英文字体
\setsansfont{Consolas} %设置英文字体 Monaco, Consolas,  Fantasque Sans Mono
%\setmainfont{Consolas} %设置英文字体

\definecolor{mygreen}{rgb}{0,0.6,0}
\definecolor{mygray}{rgb}{0.5,0.5,0.5}
\definecolor{mymauve}{rgb}{0.58,0,0.82}
\lstset{ %
    backgroundcolor=\color{white},   % choose the background color
    basicstyle=\footnotesize\ttfamily,        % size of fonts used for the code
    columns=fullflexible,
    breaklines=true,                 % automatic line breaking only at whitespace
    captionpos=b,                    % sets the caption-position to bottom
    tabsize=4,
    backgroundcolor=\color[RGB]{245,245,244},            % 设定背景颜色
    commentstyle=\color{mygreen},    % comment style
    escapeinside={\%*}{*)},          % if you want to add LaTeX within your code
    keywordstyle=\color{blue},       % keyword style
    stringstyle=\color{mymauve}\ttfamily,     % string literal style
    showstringspaces=false,                % 不显示字符串中的空格
    frame=none,
    rulesepcolor=\color{red!20!green!20!blue!20},
    % identifierstyle=\color{red},
    language=c++,
}

% 设置hyperlink的颜色
\newcommand\myshade{85}
\colorlet{mylinkcolor}{violet}
\colorlet{mycitecolor}{YellowOrange}
\colorlet{myurlcolor}{Aquamarine}

\hypersetup{
  linkcolor  = mylinkcolor!\myshade!black,
  citecolor  = mycitecolor!\myshade!black,
  urlcolor   = myurlcolor!\myshade!black,
  colorlinks = true,
}


\title{ASDAN商业竞赛项目 API设计}
\author{Tomato小组}

\begin{document}

\maketitle

\tableofcontents

\newpage

\begin{enumerate}
	\item Client API Endpoints: 为特定用户提供相关信息。
	\item Admin API Endpoints: 为管理员提供相关信息。
	\item Administrative Endpoints: 登录和交易时使用。
	\item Utility Endpoints: 询问系统相关信息。
\end{enumerate}

\section{一些修改及说明}
10.29:修改了时间的表示方法,目前按照ISO-8601格式,具体请参照这篇文章:\url{https://stackoverflow.com/questions/19013562/java-dates-and-standard-formats}。

\begin{lstlisting}
"yyyy-MM-dd'T'HH:mm:ss.SSS'Z'"
\end{lstlisting}

\section{Client API Endpoints}

为特定用户提供相关信息。

\subsection{Get Information}
输入ID,获得与这一ID相关的用户的配置信息。用户有默认头像。如果比赛还没有开始,则rank为0。

TO-DO:用户自己设置头像。
TO-DO:三个用户同时登录。

\subsubsection*{Request}
\begin{lstlisting}
GET /api/client/info/id={id}

Host: localhost:8080
Auth:
Content-type: application/json
Accept: application/json
\end{lstlisting}
\subsubsection*{Returns}
\begin{lstlisting}
HTTP 200 OK

{
    "username": "team1",
    "id":"3",
    "avatar": "/9j/4QAYRXhpZg ....",
    "gameStatus": "第2轮"
    "rank": 1
}
\end{lstlisting}
\subsubsection*{Error}
\begin{lstlisting}
HTTP 404 NOT FOUND
\end{lstlisting}







\subsection{Get Property}
输入ID,获得与这一ID相关的用户的财产信息。包括机器的使用情况和材料的价格。
\subsubsection*{Request}
\begin{lstlisting}
GET /api/client/property/id={id}

Host: localhost:8080
Auth:
Content-type: application/json
Accept: application/json
\end{lstlisting}

\subsubsection*{Returns}
\begin{lstlisting}
HTTP 200 OK

{
   "wealth": 3000,
   "machine":
    [
        {
            "id": "0073",
            "type": "type1",
            "left": 3
        },
        {
            "id": "0793",
            "type": "type1",
            "left": 0
        },
        {
            "id": "8765",
            "type": "type2",
            "left": 2
        }
    ],
    "material":
    [
        {
            "type": "wood",
            "price": 10,
            "number": 20,
        },
        {
           "type": "brick",
            "price": 20,
            "number": 0,
        },
        {
          "type": "cement",
            "price": 80,
            "number": 150,
        }
    ]
}
\end{lstlisting}

\subsubsection*{Error}
\begin{lstlisting}
HTTP 404 NOT FOUND
\end{lstlisting}






\subsection{Update Information}
向服务器发送更改过的用户配置信息。
\subsubsection*{Request}
\begin{lstlisting}
PUT /api/client/info/id={id}

Host: localhost:8080
Auth:
Content-type: application/json
Accept: application/json

{
    "username": "team1",
    "id":3,
    "avatar": ".....",
    "rank": 1,
}
\end{lstlisting}
subsubsection*{Returns}
\begin{lstlisting}
HTTP 200 OK

{
    "username": "team1",
    "id":3,
    "avatar": "/9j/4QAYRXhpZg ....",
    "rank": 1
}
\end{lstlisting}
\subsubsection*{Error}
\begin{lstlisting}
HTTP 404 NOT FOUND
\end{lstlisting}





\subsection{Produce}
向服务器发送 需要生产啊的机器的ID和生产次数
\subsubsection*{Request}
\begin{lstlisting}
POST /api/client/property/produce/id={id}

Host: localhost:8080
Auth:
Content-type: application/json
Accept: application/json

{
   "id":2,
   "times":1,
}
\end{lstlisting}
\subsubsection*{Returns}
\begin{lstlisting}
HTTP 200 OK

{
   "wealth":1000,
   "machine":
    [
        {
            "id": "0073",
            "type": "type1",
            "left": 3
        },
        {
            "id": "0793",
            "type": "type1",
            "left": 0
        },
        {
            "id": "8765",
            "type": "type2",
            "left": 2
        }
    ],
    "material":
    [
        {
       	   "type": "wood",
            "price": 10,
            "number": 20,
        },
        {
       	   "type": "brick",
            "price": 20,
            "number": 0,
        },
        {
       	   "type": "cement",
            "price": 80,
            "number": 150,
        }
    ]
}
\end{lstlisting}
\subsubsection*{Error}
\begin{lstlisting}
HTTP 404 NOT FOUND
\end{lstlisting}





\subsection{Sell Material}
向服务器发送 出售的Material的种类,数量,和单位数量的价格,以及接受的人的ID
\subsubsection*{Request}
\begin{lstlisting}
POST /api/client/property/sellMaterial/id={id}

Host: localhost:8080
Auth:
Content-type: application/json
Accept: application/json

{
   "type":"wood"
   "number":2
   "price":300,
   "recieverID":7
}
\end{lstlisting}
\subsubsection*{Returns}
\begin{lstlisting}
HTTP 200 OK
\end{lstlisting}

\subsubsection*{Error}
\begin{lstlisting}
HTTP 404 NOT FOUND
\end{lstlisting}









\subsection{Sell Machine}
向服务器发送: 出售的机器的id, 售出的价格,和售出的一台机器的价格, 以及接受机器的人的ID
\subsubsection*{Request}
\begin{lstlisting}
POST /api/client/property/sellMachine/id={id}

Host: localhost:8080
Auth:
Content-type: application/json
Accept: application/json

{
	"id": 9870
        "price":1000,
        "recieverID":7
}
\end{lstlisting}
\subsubsection*{Returns}
\begin{lstlisting}
HTTP 200 OK
\end{lstlisting}
\subsubsection*{Error}
\begin{lstlisting}
HTTP 404 NOT FOUND
\end{lstlisting}







\subsection{Get History}
获取与某一用户相关的历史信息。在发订单的时候客户端手动更新History。
\subsubsection*{Request}
\begin{lstlisting}
GET /api/client/history/id={id}

Host: localhost:8080
Auth:
Content-type: application/json
Accept: application/json
\end{lstlisting}
\subsubsection*{Returns}
\begin{lstlisting}
HTTP 200 OK
		[
			{
			"time":   "yyyy-MM-dd'T'HH:mm:ss.SSS'Z'",
			"target": "team1",
			"action": "sell",
			"content": "wood",
			"price": 10,
			"number": 2
			"status": 1,(完成)
			},

			{
			"time":   "yyyy-MM-dd'T'HH:mm:ss.SSS'Z'",
			"target": "team1",
			 "action": "buy",
			"content": "1234" (machine.id ==1234)
			"price": 10,
			"number": 1   	(只能是1)
			"status": 0,	(正在进行)
			},

			{
			"time":   "yyyy-MM-dd'T'HH:mm:ss.SSS'Z'",
			"target": "team1",
			"action": "buy",
			"content": "6666"    (machine.id ==1234)
			"price": 10,
			"number": 1   	(只能是1)
			"status": -1,	(失败)
			}
		]\end{lstlisting}
\subsubsection*{Error}
\begin{lstlisting}
HTTP 404 NOT FOUND
\end{lstlisting}






\section{Admin API Endpoints}
\subsection{Get All Competitions}

列出全部比赛。

\subsubsection*{Request}
\begin{lstlisting}
GET /api/admin/competition/getall

Host: localhost:8080
Auth:
Content-type: application/json
Accept: application/json
\end{lstlisting}

\subsubsection*{Returns}
\begin{lstlisting}
HTTP 200 OK

[
	{
	    "id": "competiton1_id",
	    "username": "competition1",
	    "status": "processing"
	},
	{
	    "id": "competiton2_id",
	    "username": "competition2",
	    "status": "end"
	}
]
\end{lstlisting}

\subsubsection*{Error}
\begin{lstlisting}
HTTP 204 NO CONTENT
\end{lstlisting}

\subsection{Create Competition}
新建一场比赛。注意,底层也要生成机器的id。注意每场比赛的基本配置(比赛名称,参赛人数)只能创建一次,不能修改。

\subsubsection*{Request}
\begin{lstlisting}
POST /api/admin/competition/new

Host: localhost:8080
Auth:
Content-type: application/json
Accept: application/json

{
    "username": "competition_username",
    "round": "2",
    "startWealth": "1000",
    "roundParameter":
    [
        {
            "machineStartPrice": [300, 350, 400],
            "machineNum": [1, 1, 1],
            "materialProduceCost": [10, 20, 30],
            "time": 900,
        },
        {
            "machineStartPrice": [300, 350, 400],
            "machineNum": [1, 1, 1],
            "materialProduceCost": [10, 20, 30],
            "time": 900,
        }
    ]
}
\end{lstlisting}

\subsubsection*{Returns}
\begin{lstlisting}
HTTP 201 CREATED

\end{lstlisting}

\subsubsection*{Error}
\begin{lstlisting}
HTTP 404 NOT FOUND
{
    "error":"Unable to delete. Competition with id xxx not found."
}
\end{lstlisting}


\subsection{Delete Competition By ID}

通过ID删除比赛。

\subsubsection*{Request}
\begin{lstlisting}
DELETE /api/admin/competiton/id={competition_id}

Host: localhost:8080
Auth:
Content-type: application/json
Accept: application/json
\end{lstlisting}

\subsubsection*{Returns}
\begin{lstlisting}
HTTP 200 OK
[
	{
	    "id": "competiton2_id",
	    "username": "competition2",
	    "status": "end"
	}
]
\end{lstlisting}

\subsubsection*{Error}
\begin{lstlisting}
{
    "error":"Unable to delete. Competition with id xxx not found."
}
\end{lstlisting}

\subsection{Update Competition Status}

更新比赛状态。

\subsubsection*{Request}
\begin{lstlisting}
POST /api/admin/competition/status/id={id}

Host: localhost:8080
Auth:
Content-type: application/json
Accept: application/json

{
	"status": "processing"
}
\end{lstlisting}

\subsubsection*{Returns}
\begin{lstlisting}
HTTP 200 OK
\end{lstlisting}

\subsubsection*{Error}
\begin{lstlisting}
HTTP 404 NOT FOUND
{
    "error":"Unable to update. Competition with id xxx not found"
}
\end{lstlisting}


\subsection{Get Competition Property}

从服务器按id获取某一比赛的各种属性。如果该比赛的属性尚未被设置,则该项为空。属性包括名称、比赛轮数(如果比赛已开始,则不能删除已开始或结束的轮)、比赛各项参数(不能修改已开始或结束的轮的参数)、机器的id等等。

\subsubsection*{Request}
\begin{lstlisting}
GET /api/admin/competition/property/id={id}

Host: localhost:8080
Auth:
Content-type: application/json
Accept: application/json
\end{lstlisting}

\subsubsection*{Returns}
\begin{lstlisting}
HTTP 200 OK
{
    "id": "competition_id",
    "username": "competition_username",
    "status": "not started",
    "teamNum": 1,
    "participantNum": 2,
    "team":
    [
        {
            "username": "team1",
            "participant": ["member1", "member2", "member2"],
            "password": "password",
        }
    ]
    "round": "1",
    "startWealth": "1000",
    "roundParameter":
    [
        {
            "machineStartPrice": [300, 350, 400],
            "machineNum": [1, 1, 1],
            "materialProduceCost": [10, 20, 30],
            "time": 900,
        }
    ]
}

\end{lstlisting}

\subsubsection*{Error}
\begin{lstlisting}
HTTP 404 NOT FOUND
{
    "error": "Competition with id xxx not found."
}
\end{lstlisting}

\subsection{Update Competition Property}

更新比赛的各种属性。属性包括名称、比赛轮数(如果比赛已开始,则不能更改)、比赛各项参数(不能修改已开始或结束的轮的参数)。

\subsubsection*{Request}
\begin{lstlisting}
PUT /api/admin/competition/property/id={id}

Host: localhost:8080
Auth:
Content-type: application/json
Accept: application/json

{
    "round": "2",
    "startWealth": "1000",
    "round_parameter":
    [
        {
            "machineStartPrice": [300, 350, 400],
            "machineNum": [1, 1, 1],
            "materialProduceCost": [10, 20, 30],
            "time": 900,
        },
        {
            "machineStartPrice": [300, 350, 400],
            "machineNum": [1, 1, 1],
            "materialProduceCost": [10, 20, 30],
            "time": 900,
        }
    ]
}
\end{lstlisting}

\subsubsection*{Returns}
\begin{lstlisting}
HTTP 201 CREATED
\end{lstlisting}

\subsubsection*{Error}
\begin{lstlisting}
HTTP 404 NOT FOUND
{
    "error": "Competition with id xxx not found."
}
\end{lstlisting}

\begin{lstlisting}
HTTP 400 INVALID REQUEST
{
    "error": "Cannot update competition id xxx with given changes."
}
\end{lstlisting}

\subsection{Get Competition Information}
获取当前比赛信息,包括队伍的数量、资产、交易记录、机器的使用情况等。

\subsubsection*{Request}
\begin{lstlisting}
GET /api/admin/competition/info/id={id}

Host: localhost:8080
Auth:
Content-type: application/json
Accept: application/json
\end{lstlisting}

\subsubsection*{Returns}
\begin{lstlisting}
HTTP 200 OK
{
    "id": "competition_id",
    "username": "competition_username",
    "status": "not started",
    "round": "2",
    "present_round": "0",
    "team_info":
    [
        {
            "id": "id1",
            "wealth": "100",
            "material": ["30", "40", "50"],
            "machine":
            [
                 {
                    "id": "machine1_id",
                    "type": "type1",
                    "left": "3"
                 },
                 {
                    "id": "machine2_id",
                    "type": "type2",
                    "left": "2"
                 }
            ]
        },
        {
            "id": "id2",
            "wealth": "100",
            "material": ["30", "40", "50"],
            "machine":
            [
                 {
                    "id": "machine1_id",
                    "type": "type1",
                    "left": "3"
                 },
                 {
                    "id": "machine2_id",
                    "type": "type2",
                    "left": "2"
                 }
            ]
        }
    ],
    "trade_history":
    [
        {
            "time": "yyyy-MM-dd'T'HH:mm:ss.SSS'Z'",
            "sell": "team_id1",
            "buy": "team_id2",
            "content": {"material1": 1},
            "price": "10"
        },
         {
            "time": "yyyy-MM-dd'T'HH:mm:ss.SSS'Z'",
            "sell": "team_id1",
            "buy": "team_id2",
            "content": {"material2": 1},
            "price": "20"
        }
    ]
}

\end{lstlisting}

\subsubsection*{Error}
\begin{lstlisting}
HTTP 404 NOT FOUND
{
    "error": "Competition with id 1 not found."
}
\end{lstlisting}

\subsection{Update Competition Information}
向服务器发送对比赛的更新信息。增加机器、分配财产之类的。

\subsubsection*{Request}
\begin{lstlisting}
PUT /api/admin/competition/info/id={id}

Host: localhost:8080
Auth:
Content-type: application/json
Accept: application/json

{
    "round": "2",
    "present_round": "0",
    "team_info":
    [
        {
            "id": "id1",
            "wealth": "100",
            "machine":
            [
                 {
                    "id": "machine1_id",
                    "left": "3"
                 },
                 {
                    "id": "machine2_id",
                    "left": "2"
                 }
            ]
        },
        {
            "id": "id2",
            "wealth": "100",
            "machine":
            [
                 {
                    "id": "machine1_id",
                    "left": "3"
                 },
                 {
                    "id": "machine2_id",
                    "left": "2"
                 }
            ]
        }
    ]
}
\end{lstlisting}

\subsubsection*{Returns}
\begin{lstlisting}
HTTP 200 OK
{
    "id": "competition_id",
    "username": "competition_username",
    "status": "not started",
    "round": "2",
    "present_round": "0",
    "team_info":
    [
        {
            "id": "id1",
            "wealth": "100",
            "material": ["30", "40", "50"],
            "machine":
            [
                 {
                    "id": "machine1_id",
                    "type": "type1",
                    "left": "3"
                 },
                 {
                    "id": "machine2_id",
                    "type": "type2",
                    "left": "2"
                 }
            ]
        },
        {
            "id": "id2",
            "wealth": "100",
            "material": ["30", "40", "50"],
            "machine":
            [
                 {
                    "id": "machine1_id",
                    "type": "type1",
                    "left": "3"
                 },
                 {
                    "id": "machine2_id",
                    "type": "type2",
                    "left": "2"
                 }
            ]
        }
    ],
    "trade_history":
    [
        {
            "time": "hh:MM:ss",
            "sell": "team_id1",
            "buy": "team_id2",
            "content": {"material1": 1},
            "price": "10"
        },
         {
            "time": "hh:MM:ss",
            "sell": "team_id1",
            "buy": "team_id2",
            "content": {"material2": 1},
            "price": "20"
        }
    ]
}

\end{lstlisting}

\subsubsection*{Error}
\begin{lstlisting}
HTTP 404 NOT FOUND
{
    "error": "Competition with id xxx not found."
}
\end{lstlisting}

\begin{lstlisting}
HTTP 400 INVALID REQUEST
{
    "error": "Cannot update competition id xxx with given information."
}
\end{lstlisting}

\section{Administrative Endpoints}
登录和交易时使用。

\subsection{Login Admin}
管理员登录。

\subsubsection*{Request}
\begin{lstlisting}
POST /api/admin/login

Host: localhost:8080
Auth:
Content-type: application/json
Accept: application/json

{
    "username": "admin",
    "password": "admin",
}
\end{lstlisting}

\subsubsection*{Returns}
\begin{lstlisting}
HTTP 200 OK
{
    "username": "admin",
    "token": "1283091828021803120",
}

\end{lstlisting}

\subsubsection*{Error}
\begin{lstlisting}
HTTP 401 NOTAUTHORIZED
{
    "error": "Admin with username admin doesn\'t exist or password is wrong."
}
\end{lstlisting}

\subsection{Login Client}
用户登录。

\subsubsection*{Request}
\begin{lstlisting}
POST /api/client/login

Host: localhost:8080
Auth:
Content-type: application/json
Accept: application/json

{
    "username": "client",
    "id": "3",
    "password": "client",
}
\end{lstlisting}

\subsubsection*{Returns}
\begin{lstlisting}
HTTP 200 OK
{
    "username": "client",
    "token": "1283091828021803120",
}

\end{lstlisting}

\subsubsection*{Error}
\begin{lstlisting}
HTTP 401 NOTAUTHORIZED
{
    "error": "Client with userusername admin doesn\'t exist or password is wrong."
}
\end{lstlisting}

\section{Utility Endpoints}
询问系统相关信息。

\end{document}
