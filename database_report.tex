% !Mode:: "TeX:UTF-8"
\documentclass{article}
\usepackage[hyperref, UTF8]{ctex}
\usepackage[dvipsnames]{xcolor}
\usepackage{geometry}
\usepackage{amsmath}
\usepackage{amsfonts}
\usepackage[section]{placeins}
\usepackage{listings}
\usepackage{pgfplotstable}
\usepackage{pgfplots}
\usepackage{fontspec}
\usepackage{comment}
\usepackage{booktabs} % 表格上的不同横线
\setmonofont[Mapping={}]{Consolas}	%英文引号之类的正常显示,相当于设置英文字体
\setsansfont{Consolas} %设置英文字体 Monaco, Consolas,  Fantasque Sans Mono
\setmainfont{Consolas} %设置英文字体

\definecolor{mygreen}{rgb}{0,0.6,0}
\definecolor{mygray}{rgb}{0.5,0.5,0.5}
\definecolor{mymauve}{rgb}{0.58,0,0.82}
\lstset{ %
    backgroundcolor=\color{white},   % choose the background color
    basicstyle=\footnotesize\ttfamily,        % size of fonts used for the code
    columns=fullflexible,
    breaklines=true,                 % automatic line breaking only at whitespace
    captionpos=b,                    % sets the caption-position to bottom
    tabsize=4,
    backgroundcolor=\color[RGB]{245,245,244},            % 设定背景颜色
    commentstyle=\color{mygreen},    % comment style
    escapeinside={\%*}{*)},          % if you want to add LaTeX within your code
    keywordstyle=\color{blue},       % keyword style
    stringstyle=\color{mymauve}\ttfamily,     % string literal style
    showstringspaces=false,                % 不显示字符串中的空格
    frame=none,
    rulesepcolor=\color{red!20!green!20!blue!20},
    % identifierstyle=\color{red},
    language=sql,
}

% 设置hyperlink的颜色
\newcommand\myshade{85}
\colorlet{mylinkcolor}{violet}
\colorlet{mycitecolor}{YellowOrange}
\colorlet{myurlcolor}{Aquamarine}

\hypersetup{
  linkcolor  = mylinkcolor!\myshade!black,
  citecolor  = mycitecolor!\myshade!black,
  urlcolor   = myurlcolor!\myshade!black,
  colorlinks = true,
}

\renewcommand\figurename{图}

\setcounter{secnumdepth}{5}

\title{软件工程大作业文档}
\author{Tomato}
\date{2018年1月}

\begin{document}

\maketitle

\section{数据库(顶层)设计说明}
\label{数据库(顶层)设计说明}
\subsection{引言}
\subsubsection{标识}
\begin{comment}
本条应包含本文档适用的数据库的完整标识,(若适用)包括标识号、标题、缩略词语、版本号、发行号。
\end{comment}
数据库版本为1.0.0。

\subsubsection{数据库概述}
\begin{comment}
本条应简述本文档适用的数据库的用途。它应描述数据库的一般性质;概括它的开发、使用和维护的历史;标识项目的投资方、需方、用户、开发方和支持机构;标识当前和计划的运行现场;并列出其他有关文档。
\end{comment}
数据库的用途是存放必要的用户管理数据和比赛数据。项目的用户是ASDAN商业竞赛,开发方是Tomato团队。

\subsubsection{文档概述}
\begin{comment}
本条应概括本文档的用途与内容,并描述与其使用有关的保密性或私密性要求。
\end{comment}
本文档的用途是介绍数据库。

\subsection{引用文件}
\begin{comment}
本章应列出本文档引用的所有文档的编号、标题、修订版本和日期。也应标识不能通过正常的供货渠道获得的所有文档的来源。
\end{comment}
没有引用别的文件。

\subsection{数据库级测试决策}
\begin{comment}
本章应根据需要分条给出数据库级设计决策,即数据库行为设计决策(从用户的角度看,该数据库如何满足它的需求而忽略内部实现)和其他影响数据库进一步设计的决策。如果所有这些决策在系统或CSCI需求中均是明确的,本章应如实陈述。对应于指定为关键性需求(如安全性、保密性、私密性需求)的设计决策,应在单独的条中加以描述。如果设计决策依赖于系统状态或方式,则应指出这种依赖性。如果设计决策的部分或全部已在定制的或商用的数据库管理系统(DBMS)的文档中作了描述,本章可引用它们。应给出或引用理解设计所需的设计约定。数据库级设计决策的例子如下:
a.关于该数据库应接受的查询或其他输入和它应产生的输出(显示、报告、消息、响应等)的设计决策,包括与其他系统、HWCI,CSCI和用户的接口(本文的5.x.d 标识了本说明要考虑的主题)。如果该信息的部分或全部已在接口设计说明(IDD)中给出,此处可引用。
b.有关响应每次输入或查询的数据库行为的设计决策,包括动作、响应时间和其他性能特性、所选择的方程式/算法/规则、配置和对不允许的输入的处理。
c.有关数据库/数据文件如何呈现给用户的设计决策。(本文的4.x标识了本说明要考虑的主题).
d.有关要使用什么数据库管理系统(包括名字、版本/发行)的设计决策和为适应需求的变化而引人到数据库内部的灵活性类型的设计决策。
e.有关数据库要提供的可用性、保密性、私密性和运行连续性的层次与类型的设计决策。
f.有关数据库的分布(如客户机/服务器)、主数据库文件更新与维护的设计决策,包括一致性的维护、同步的建立/重建与维护、完整性与业务规则的实施等。
g.有关备份与恢复的设计决策,包括数据与处理分布策略、备份与恢复期间所允许的动作、对例如音像等新技术或非标准技术的特殊考虑。
h.有关重组、排序、索引、同步与一致性的设计决策,包括自动的盘管理与空间回收、优化策略、存储与空间大小、数据库内容的填充与历史数据的捕获等方面的考虑。
\end{comment}

\subsection{数据库详细设计}
\begin{comment}
本章应根据需要分条描述数据库的详细设计。设计级别数以及每一级别的名称应基于所用的设计方法学。数据库设计级别的例子包括:概念设计、内部设计、逻辑设计和物理设计。如果这些设计决策的部分或全部依赖于系统状态或方式,则应指出这种依赖性。应给出或引用为理解这些设计所需的设计约定。
注:本文用术语“数据元素集合体”表示在给定的设计级别(如概念设计、内部设计、逻辑设计、物理设计)中具有结构特征(数据元素的编号/次序/分组)的任何实体、关系、模式、字段、表、数组等;同时用术语“数据元素”表示在给定的级别中没有结构特征的关系、属性、字段、配置项、数据元素等。
4.x(数据库设计级别的名称)
本条应标识一个数据库设计级别,并用所选择的设计方法的术语描述数据库的数据元素和数据元素集合体。(若适用)描述信息应包括以下内容,它们可按适合于要提供的信息的任何次序给出。
a.数据库设计中的单个数据元素特性,例如:
1)名称/标识符;
a)项目唯一标识符;
b)非技术(自然语言)名称;
c)标准数据元素名;
d)技术名称(如数据库中的字段名);
e)缩写名或同义名;
2)数据类型(字母数字、整数等);
3)大小与格式(例如字符串的长度与标点符号);
4)计量单位(如米、元、纳秒等);
5)范围或可能值的枚举(如0-99);
6)准确度(正确程度)与精度(有效位数);
7)优先级、时序、频率、容量、序列和其他约束,如数据元素是否可被更新,业务规则是否适用;
8)保密性与私密性约束;
9)来源(设置/发送实体)与接收者(使用/接收实体)。
b.数据库设计中的数据元素集合体(记录、消息、文件、数组、显示、报表等)的特性,例如:
1)名称/标识符;
a)项目唯一标识符;
b)非技术(自然语言)名称;
c)技术名称(如代码或数据库中的记录或数据结构名);
d)缩写名或同义名;
2)数据元素集合体中的数据元素及其结构(编号、次序、分组);
3)媒体(如盘)及媒体上数据元素/集合体的结构;
4)显示和其他输出的视听特性(如颜色、布局、字体、图标及其他显示元素、蜂鸣声、亮度等);
5)数据元素集合体之间的关系,如排序/访问特性;
6)优先级、时序、频率、容量、序列和其他约束,如数据集合体是否可被更新,业务规则是否适用;
7)保密性与私密性约束;
8)来源(设置/发送实体)与接收者(使用/接收实体)。
\end{comment}

\subsubsection{概念设计}

\subsubsection{内部设计}

\subsubsection{逻辑设计}
数据库的操作实体类共有7个,接下来将详细介绍每个类内部的成员变量。

\paragraph{Admin}
存放管理员的类。
\begin{description}
  \item[serialVersionUID (static final long)] 用于序列化不同版本的类。
  \item[adminId (Long)] 主键,存放管理员ID。
  \item[name (String)] 管理员名称。
  \item[username (String)] 管理员用户名。
  \item[password (String)] 管理员密码。
  \item[enabled (boolean)] 管理员权限是否被启用。
  \item[role (String)] 当前账户的性质,管理员为"ADMIN"。
  \item[competitionList (List<Competition>)] 当前管理员管理的全部比赛。
\end{description}

\paragraph{Competition}
存放比赛的类。
\begin{description}
  \item[serialVersionUID (static final long)] 用于序列化不同版本的类。
  \item[competitionId (Long)] 主键,存放比赛ID。
  \item[name (String)] 比赛名称。
  \item[round (int)] 比赛一共有多少轮。
  \item[presentRound (int)] 比赛当前进行到了第几轮。
  \item[roundList (List<Round>)] 比赛每一轮的信息。
  \item[teamList (List<Team>)] 比赛中每支队伍的信息。
  \item[produceList (List<Produce>)] 比赛中全部生产的信息。
  \item[competitionTradeList (List<Trade>)] 比赛中全部交易的信息。
  \item[initial (int)] 比赛队伍的初始资产。
  \item[status (Competition::Status)] 比赛状态,分为NOT\_START(比赛未开始)、AUCTION\_NOT\_RECORDED(拍卖结果未记录)、AUCTION\_RECORDED (拍卖结果已记录)、TRADE(交易中)、REST(休息中)和END(比赛已结束)6种。
\end{description}

\paragraph{Competition::Round}
存放比赛中一轮的信息的类。
\begin{description}
  \item[roundId (Long)] 主键,存放轮ID。
  \item[startTime (Date)] 本轮开始时间。
  \item[time (Integer)] 本轮持续时间,以分钟为单位。
  \item[machineNumberMap (Map<Material, Integer>)] 每种机器的个数。
  \item[machineStartPriceMap (Map<Material, Integer>)] 每种机器的起拍价格。
  \item[machineList (List<Machine>)] 本轮拍卖的全部机器。
  \item[machineAuctionPriceMap (Map<Machine, Integer>)] 每个机器的拍卖价格。
  \item[materialPriceMap (Map<Material, Integer>)] 每种材料的生产成本。
  \item[produceList (List<Produce>)] 本轮生产材料的列表。
  \item[tradeList (List<Trade>)] 本轮的交易列表。
  \item[materialPerHouse (Map<Material, Integer>)] 本轮建造一栋房子所需的材料数量。
\end{description}

\paragraph{Machine}
存放机器的类。
\begin{description}
  \item[serialVersionUID (static final long)] 用于序列化不同版本的类。
  \item[machineId (Long)] 主键,存放机器ID。
  \item[owner (Team)] 机器的拥有者。
  \item[material (Material)] 机器生产的材料种类。
  \item[left (Integer)] 机器剩余的生产次数。
  \item[tradeList (List<Trade>)] 这台机器对应的交易列表。
\end{description}

\paragraph{Produce}
存放交易的类。
\begin{description}
  \item[serialVersionUID (static final long)] 用于序列化不同版本的类。
  \item[produceId (Long)] 主键,存放交易ID。
  \item[team (Team)] 进行生产的队伍。
  \item[machine (Machine)] 进行生产用的机器。
  \item[material (Material)] 生产的材料类型。
  \item[amount (Integer)] 生产材料的数量。
  \item[price (Integer)] 生产单位材料的成本。
  \item[produceTime (Date)] 生产时间。
\end{description}

\paragraph{Team}
存放队伍信息的类。
\begin{description}
  \item[serialVersionUID (static final long)] 用于序列化不同版本的类。
  \item[teamId (Long)] 主键,存放队伍ID。
  \item[name (String)] 队伍名称。
  \item[username (String)] 队伍用户名。
  \item[password (String)] 队伍密码。
  \item[picture (CustomMultipartFile)] 队伍头像。
  \item[rank (Integer)] 队伍当前排名。
  \item[machineList (List<Long>)] 队伍拥有的所有机器的ID。
  \item[tradeList (List<Long>)] 队伍进行的全部交易的ID。
  \item[tramProduceList (List<Long>)] 队伍进行的全部生产的ID。
  \item[money (Integer)] 队伍当前现金数量。
  \item[materialMap (Map<Material, Integer>)] 队伍当前拥有各种材料的数量。
  \item[materialLockMap (Map<Material, Boolean>)] 某种材料是否属于出售状态。
  \item[machineLockMap (Map<Machine, Boolean>)] 队伍拥有的某个机器是否处于出售状态。
  \item[enabled (boolean)] 当前队伍的权限是否被启用。
  \item[member (List<String>)] 队伍成员列表。
  \item[competition (Competition)] 队伍所在的比赛。
\end{description}

\paragraph{Trade}
存放交易信息的类。
\begin{description}
  \item[serialVersionUID (static final long)] 用于序列化不同版本的类。
  \item[tradeId (Long)] 主键,存放交易ID。
  \item[type (TradeType)] 交易类型,分为MATERIAL(交易材料)、MACHINE(交易机器)、AUCTION(拍卖)三类。
  \item[seller (Team)] 交易的卖方队伍,如果交易类型为拍卖,则置为空。
  \item[buyer (Team)] 交易的买方队伍。
  \item[material (Material)] 交易的材料类型,如果交易类型为拍卖或交易机器,则置为空。
  \item[machine (Machine)] 交易的机器,如果交易类型为交易材料,则置为空。
  \item[amount (Integer)] 交易的材料/机器数量,如果交易的为机器,则置为1。
  \item[price (Integer)] 交易的材料/机器的单价。
  \item[proposeTime (Date)] 卖方提出交易的时间。
  \item[tradeTime (Date)] 交易成功的时间。
  \item[status (Status)] 交易状态,分为PENDING(等待买方确认中)、CANCELLED(已取消)和SUCCEED(已成功)三种。
\end{description}

\subsubsection{物理设计}
根据上述实体类设计,Hibernate一共生成了29张表,下面列出每张表的每一列:
\small{
\begin{enumerate}
  \item admin
    \begin{description}
      \item [admin\_id] BIGINT(20), AUTO INCREMENT, PRIMARY KEY
      \item [enabled] INT(11)
      \item [name] VARCHAR(255)
      \item [password] VARCHAR(255)
      \item [role] VARCHAR(5)
      \item [username] VARCHAR(255)
    \end{description}
  \item admin\_competition\_list
    \begin{description}
      \item[admin\_admin\_id] BIGINT(20), FOREIGN KEY(admin.admin\_id)
      \item[competition\_list\_competition\_id] BIGINT(20), UNIQUE, FOREIGN KEY(competition\_id.competition\_id)
    \end{description}
  \item competition\$round\_machine\_auction\_price\_map
    \begin{description}
      \item[competition\$round\_round\_id] BIGINT(20), PRIMARY KEY, FOREIGN KEY(round.round\_id)
      \item[machine\_auction\_pricemap] INT(11)
      \item[machine\_auction\_price\_map\_key] BIGINT(20), PRIMARY KEY, FOREIGN KEY(machine.machine\_id)
    \end{description}
  \item competition\$round\_machine\_number\_map
    \begin{description}
      \item[competition\$round\_round\_id] BIGINT(20), PRIMARY KEY, FOREIGN KEY(round.round\_id)
      \item[machine\_number\_map] INT(11)
      \item[machine\_number\_map\_key] INT(11), PRIMARY KEY
    \end{description}
  \item competition\$round\_machine\_start\_price\_map
    \begin{description}
      \item[competition\$round\_round\_id] BIGINT(20), PRIMARY KEY, FOREIGN KEY(round.round\_id)
      \item[machine\_start\_price\_map] INT(11)
      \item[machine\_start\_price\_map\_key] INT(11), PRIMARY KEY
    \end{description}
  \item competition\$round\_material\_per\_house
    \begin{description}
      \item[competition\$round\_round\_id] BIGINT(20), PRIMARY KEY, FOREIGN KEY(round.round\_id)
      \item[material\_per\_house] INT(11)
      \item[material\_per\_house\_key] INT(11), PRIMARY KEY
    \end{description}
  \item competition\$round\_material\_price\_map
    \begin{description}
      \item[competition\$round\_round\_id] BIGINT(20), PRIMARY KEY, FOREIGN KEY(round.round\_id)
      \item[material\_price\_map] INT(11)
      \item[material\_price\_map\_key] INT(11), PRIMARY KEY
    \end{description}
  \item competition\_id
    \begin{description}
      \item[competition\_id] BIGINT(20), AUTO INCREMENT, PRIMARY KEY
      \item[initial] INT(11)
      \item[name] VARCHAR(255)
      \item[present\_round] INT(11)
      \item[round] INT(11)
      \item[status] INT(11)
    \end{description}
  \item competition\_id\_produce\_list
    \begin{description}
      \item[competition\_competition\_id] BIGINT(20), UNIQUE, FOREIGN KEY(competition\_id.competition\_id)
      \item[produce\_list\_produce\_id] BIGINT(20), UNIQUE, FOREIGN KEY(produce.produce\_id)
    \end{description}
  \item competition\_id\_round\_list
    \begin{description}
      \item[competition\_competition\_id] BIGINT(20), UNIQUE, FOREIGN KEY(competition\_id.competition\_id)
      \item[round\_list\_round\_id] BIGINT(20), UNIQUE, FOREIGN KEY(round.round\_id)
    \end{description}
  \item competition\_id\_team\_list
    \begin{description}
      \item[competition\_competition\_id] BIGINT(20), UNIQUE, FOREIGN KEY(competition\_id.competition\_id)
      \item[team\_list\_team\_id] BIGINT(20), UNIQUE, FOREIGN KEY(team.team\_id)
    \end{description}
  \item competition\_id\_trade\_list
    \begin{description}
      \item[competition\_competition\_id] BIGINT(20), UNIQUE, FOREIGN KEY(competition\_id.competition\_id)
      \item[trade\_list\_trade\_id] BIGINT(20), UNIQUE, FOREIGN KEY(trade.trade\_id)
    \end{description}
  \item custom\_multipart\_file
    \begin{description}
      \item[file\_id] BIGINT(20), AUTO INCREMENT, PRIMARY KEY
      \item[content\_type] VARCHAR(255)
      \item[file\_original\_size] BIGINT(20)
      \item[img\_content] longblob
    \end{description}
  \item machine
    \begin{description}
      \item[machine\_id] BIGINT(20), AUTO INCREMENT, PRIMARY KEY
      \item[left\_amount] INT(11)
      \item[material] INT(11)
      \item[team\_id] BIGINT(20), FOREIGN KEY(team.team\_id)
    \end{description}
  \item machine\_trade\_list
    \begin{description}
      \item[machine\_id] BIGINT(20), AUTO INCREMENT, PRIMARY KEY, FOREIGN KEY(machine.machine\_id)
      \item[trade\_list\_trade\_id] BIGINT(20), UNIQUE, FOREIGN KEY(trade.trade\_id)
    \end{description}
  \item produce
    \begin{description}
      \item[produce\_id] BIGINT(20), AUTO INCREMENT, PRIMARY KEY
      \item[amount] INT(11)
      \item[material] INT(11)
      \item[price] INT(11)
      \item[producetime] DATETIME
      \item[machine] BIGINT(20), FOREIGN KEY(machine.machine\_id)
      \item[team] BIGINT(20), FOREIGN KEY(team.team\_id)
    \end{description}
  \item round
    \begin{description}
      \item[round\_id] BIGINT(20), AUTO INCREMENT, PRIMARY KEY
      \item[start\_time] DATETIME
      \item[time] INT(11)
    \end{description}
  \item round\_machine\_list
    \begin{description}
      \item[competition\$round\_round\_id] BIGINT(20), FOREIGN KEY(round.round\_id)
      \item[machine\_list\_machine\_id] BIGINT(20), UNIQUE, FOREIGN KEY(machine.machine\_id)
    \end{description}
  \item round\_produce\_list
    \begin{description}
      \item[competition\$round\_round\_id] BIGINT(20), FOREIGN KEY(round.round\_id)
      \item[produce\_list\_produce\_id] BIGINT(20), UNIQUE, FOREIGN KEY(produce.produce\_id)
    \end{description}
  \item round\_trade\_list
    \begin{description}
      \item[competition\$round\_round\_id] BIGINT(20), FOREIGN KEY(round.round\_id)
      \item[trade\_list\_trade\_id] BIGINT(20), UNIQUE, FOREIGN KEY(trade.trade\_id)
    \end{description}
  \item team
    \begin{description}
      \item[team\_id] BIGINT(20), AUTO INCREMENT, PRIMARY KEY
      \item[enabled] INT(11)
      \item[money] INT(11)
      \item[name] VARCHAR(255)
      \item[password] VARCHAR(255)
      \item[rank] INT(11)
      \item[username] VARCHAR(255)
      \item[competition\_id] BIGINT(20), FOREIGN KEY(competition\_id.competition\_id)
      \item[picture] BIGINT(20), FOREIGN KEY(custom\_multipart\_file.file\_id)
    \end{description}
  \item team\_machine\_list
    \begin{description}
      \item[team\_team\_id] BIGINT(20), FOREIGN KEY(team.team\_id)
      \item[machine\_list\_machine\_id] BIGINT(20), UNIQUE, FOREIGN KEY(machine.machine\_id)
    \end{description}
  \item team\_machine\_lock\_map
    \begin{description}
      \item[team\_team\_id] BIGINT(20), PRIMARY KEY, FOREIGN KEY(team.team\_id)
      \item[machine\_lock\_map] BIT(1)
      \item[machine\_lock\_map\_key] BIGINT(20), PRIMARY KEY
    \end{description}
  \item team\_material\_lock\_map
    \begin{description}
      \item[team\_team\_id] BIGINT(20), PRIMARY KEY, FOREIGN KEY(team.team\_id)
      \item[material\_lock\_map] BIT(1)
      \item[material\_lock\_map\_key] BIGINT(20), PRIMARY KEY
    \end{description}
  \item team\_material\_map
    \begin{description}
      \item[team\_team\_id] BIGINT(20), PRIMARY KEY, FOREIGN KEY(team.team\_id)
      \item[material\_map] BIT(1)
      \item[material\_map\_key] BIGINT(20), PRIMARY KEY
    \end{description}
  \item team\_member
    \begin{description}
      \item[team\_team\_id] BIGINT(20), PRIMARY KEY, FOREIGN KEY(team.team\_id)
      \item[member] VARCHAR(255)
    \end{description}
  \item team\_team\_produce\_list
    \begin{description}
      \item[team\_team\_id] BIGINT(20), PRIMARY KEY, FOREIGN KEY(team.team\_id)
      \item[team\_produce\_list] VARCHAR(255)
    \end{description}
  \item team\_trade\_list
    \begin{description}
      \item[team\_team\_id] BIGINT(20), PRIMARY KEY, FOREIGN KEY(team.team\_id)
      \item[trade\_list\_trade\_id] BIGINT(20), UNIQUE, FOREIGN KEY(trade.trade\_id)
    \end{description}
  \item trade
    \begin{description}
      \item[trade\_id] BIGINT(20), AUTO INCREMENT, PRIMARY KEY
      \item[amount] INT(11)
      \item[material] INT(11)
      \item[price] INT(11)
      \item[proposetime] DATETIME
      \item[status] INT(11)
      \item[tradetime] DATETIME
      \item[type] INT(11)
      \item[buyer\_id] BIGINT(20), FOREIGN KEY(team.team\_id)
      \item[machine\_id] BIGINT(20), FOREIGN KEY(machine.machine\_id)
      \item[seller\_id] BIGINT(20), FOREIGN KEY(team.team\_id)
    \end{description}
\end{enumerate}
} % small

\subsection{用于数据库访问或操纵的软件配置项的详细设计}
\begin{comment}
本章应分条描述用于数据库访问或操纵的每个软件配置项。如果该信息的部分或全部已在别处提供,如在软件(结构)设计说明(SDD)、定制的DBMS的SDD,商用的DBMS的用户手册等处,在此可引用该信息,而无需重复说明。如果设计的部分或全部依赖于系统状态或方式,则应指出这种依赖性。如果该设计信息在多条中出现,则可只描述一次,而在其他条引用。应给出或引用为理解设计所需的设计约定。
5.x(软件配置项的项目唯一标识符或软件配置项组的指定符)
本条应用项目唯一标识符标识软件配置项并描述它。(若适用)描述应包括以下信息。作为一种变通.本条也可以指定一组软件配置项,并分条标识和描述它们。包含其他软件配置项的软件配置项可以引用那些软件配置项的说明,而无需在此重复。
a.(若有)配置项设计决策,诸如(如果以前未选)要使用的算法;
b.软件配置项设计中的约束、限制或非常规特征;
c.如果要使用的编程语言不同于该CSCI所指定的语言,应该指出,并说明使用它的理由;
d.如果软件配置项由过程式命令组成或包含过程式命令(如数据库管理系统(DBMS)中用于定义表单与报表的菜单选择、用于数据库访问与操纵的联机DBMS查询、用于自动代码生成的图形用户接口(GUI)构造器的输入、操作系统的命令或命令解释程序(shell)脚本),应有过程式命令列表和对解释它们的用户手册或其他文档的引用;
e.如果软件配置项包含、接收或输出数据,(若适用)应有对其输入、输出和其他数据元素以及数据元素集合体的说明。(若适用)本文的4.x.6提供要包含主题的列表。软件配置项的局部数据应与软件配置项的输入或输出数据分开来描述。如果该软件配置项是一个数据库,应引用相应的数据库(顶层)设计说明(DBDD):接口特性可在此处提供,也可引用相应接口设计说明。如果一给定的接口实体本文没有提及(例如,一个外部系统),但是其接口特性需要在本DBDD描述的接口实体时提到,则这些特性应以假设、或“当[未提及实体]这样做时,[软件配置项]将……”的形式描述。本条可引用其他文档(例如数据字典、协议标准、用户接口标准)代替本条的描述信息。本设计说明应包括以下内容,(若适用)它们可按适合于要提供的信息的任何次序给出,并且应从接口实体角度指出这些特性之间的区别(例如数据元素的大小、频率等)。
1)接口的项目唯一标识符;
2)(若适用)用名字、编号、版本和文档引用来标识接口实体(软件配置项、配置项、用户等);
3)由接口实体分配给接口的优先级;
4)要实现的接口的类型(例如实时数据传输、数据的存储与检索等);
5)接口实体将提供、存储、发送、访问、接收的单个数据元素的特性。本文档4.x.a标识了要提及的主题;
6)接口实体将提供、存储、发送、访问、接收的数据元素集合体(记录、消息、文件、数组、显示、报表等)的特性。本文档的4.x.6标识了要提及的主题;
7)接口实体为该接口使用通信方法的特性,例如:
a)项目唯一标识符;
b)通信链路/带宽/频率/媒体及其特性;
c)消息格式化;
d)流控制(如序列编号与缓冲区分配);
e)数据传输率、周期或非周期和传送间隔;
f)路由、寻址及命名约定;
g)传输服务,包括优先级与等级;
h)安全性/保密性/私密性考虑,如加密、用户鉴别、隔离、审核等。
8)接口实体为该接口使用协议的特性,例如:
a)项目唯一标识符;
b)协议的优先级/层次;
c)分组,包括分段与重组、路由及寻址;
d)合法性检查、错误控制、恢复过程;
e)同步,包括连接的建立、维护、终止;
f)状态、标识和其他报告特征。
9)其他特性,如接口实体的物理兼容性(尺寸、容量、负荷、电压、接插件的兼容性等);
f.如果软件配置项包含逻辑,给出其要使用的逻辑,(若适用)包括:
1)该软件配置项执行启动时,其内部起作用的条件:
2)把控制交给其他软件配置项的条件;
3)对每个输入的响应及响应时间,包括数据转换、重命名和数据传送操作;
4)该软件配置项运行期间的操作序列和动态控制序列,包括:
a)序列控制方法;
b)该方法的逻辑与输入条件,如计时偏差、优先级赋值;
c)数据在内存中的进出;
d)离散输入信号的读出,以及在软件配置项内中断操作之间的时序关系;
5)异常与错误处理。
\end{comment}

\subsection{需求的可追踪性}

\end{document}
